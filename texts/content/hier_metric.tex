%!TEX root = ../main.tex
%!TEX encoding = UTF-8
%!TEX spellcheck = en_US
%!TEX program = pdflatex

\section{Triplet-based Metrics}\label{sec:hierarchy_metrics}
Two established MSA metrics for comparing hierarchies exist: the T-measure~\cite{DBLP:conf/ismir/McFeeNB15} which compares two sets of hierarchical boundaries, and the L-measure~\cite{McfeeNFB17} which also accounts for segment labeling.
Both metrics are implemented in \texttt{mir\_eval}~\cite{DBLP:conf/ismir/RaffelMHSNLE14} and have been used in many recent works on hierarchical MSA~\cite{DBLP:journals/taslp/BuissonMEC24, DBLP:conf/icassp/TralieM19, DBLP:journals/tismir/BerardinisVCC20, DBLP:conf/apsipa/ChenSY23}.

In this section, we provide a unifying overview for both measures, as they share the same core idea of recalling sets of triplets of time points $(t,u,v)$.
We begin by showing how L-measure and different variations of the T-measure can be thought of as specialized forms of L-measure.
% We do this in the hope that it will provide a better understanding of how the metrics work and better intuition about interpreting their results.

\subsection{Discrete set formulation}
We begin by laying down the notational framework by following prior literature~\cite{DBLP:journals/tismir/NietoMWSSGM20,DBLP:journals/tismir/KinnairdM21}.

For a piece of music with time span \(T = [t_0, t_1]\), a flat segmentation has a label mapping $S(t)$:
\begin{equation}
    S(t): T \to \{y_1, y_2, \ldots\}.
    \label{eq:S}
\end{equation}
A $k$-level hierarchy is then a sequence of progressively finer segmentations
\(H = \bigl(S_{1}, S_{2}, \dots, S_k\bigr)\).

Given a hierarchy $H$, L-measure defines a relevance mapping $M$ between time point pairs $T \times T$ as the maximum level of the hierarchy at which the two time points share the same label:
\begin{equation}
    M_H(u, v) := \max \big\{k \, | \, S_k(u) = S_k(v)\big\}
    \label{eq:relevance_mapping}
\end{equation}
This is called the meet matrix of two hierarchical annotations in prior literature.

% Let's visualize some of these concepts. Show hierarchy H, M(u, v|H), and the orientation map $I_H(t,u,v)$ at a given time $t$.

Triplet based metrics are interested in the set \(\mathcal{A}(H)\) consisting of time-triplets \((t,u,v)\) where the time instances $t$ and $u$ are deemed more relevant than $t$ and $v$.
L-measure defines its significant set of triplets as:
\begin{equation}
    \mathcal{A}(H) = \{(t,u,v)|M_H(t,u) > M_H(t,v)\}
\end{equation}

Precision and recall scores are then defined by counting the proportions of triplets shared between the derived sets.
For an estimated hierarchy $\hat{H}$ and a reference $H$, we define:
\begin{align}
\mathrm{L}_{\mathrm{recall}}(\hat{H} \mid H )
&= \frac{\bigl|\mathcal{A}(\hat{H}) \,\cap\, \mathcal{A}(H)\bigr|}
    {\bigl|\mathcal{A}(H)\bigr|}, \nonumber \\
\mathrm{L}_{\mathrm{precision}}(\hat{H} \mid H ) \label{eq:lmeasure}
&= \frac{\bigl|\mathcal{A}(\hat{H}) \,\cap\, \mathcal{A}(H)\bigr|}
    {\bigl|\mathcal{A}(\hat{H})\bigr|}.
\end{align}

In the discrete set formulation, $|\mathcal{A}(H)|$ is the cardinality of a set of triplets. 
This necessitates a frame-based sampling of time triplets for metric calculation, which is computationally expensive for long sequences.
Furthermore, this sampling-based approach is an approximation that depends on time resolution, which by convention is set to 0.1 seconds.

 
\subsection{Continuous set Formulation}
We can also define the triplet recall metric in continuous time.
The set-based definition of hierarchy metrics could be expressed in terms of areas or volumes in space, which allows us to avoid the frame-based sampling of triplets.
Because $S_k(t)$ is piecewise constant, the integrals involved can all be computed in closed form, with complexity related to the number of segments in the hierarchy rather than the number of frames.

Triplet-based metrics differ only in how they define the pair relevance mapping \(M_H(u,v)\) and the significant triplet indicator \(I_H(t,u,v)\), which for L-measure is defined as:
\begin{equation}
    I_H(t,u,v) = \big[M_H(t,u) > M_H(t,v)\big]_{\mathbbm{1}}, 
    \label{eq:significant_triplet}
\end{equation}
where \([\,\cdot\,]_{\mathbbm{1}}\) is the indicator function that returns 1 if the condition is true and 0 otherwise.

Using the significant triplet indicator $I_H(t,u,v)$, we can calculate the structural richness at time $t\in T$ as a measure.
\emph{Structural information measure} is defined for each time point $t$ as:
\begin{equation}
    \iota(H;t) = \int_{T^2} I_H(t,u,v)\,\dif(u,v).
    \label{eq:local_richness}
\end{equation}
This measure captures the amount of structural information present at time \(t\) in the hierarchy \(H\).

As opposed to integrating over the $(t,u,v) \in T \times T \times T$ directly, we first integrate over the time pairs \((u,v) \in T \times T\) to obtain measures at \(t\), before aggregating over $t \in T$, as is done in the original implementation of the metric.
Doing this would mean each time point \(t\) contributes equally to the overall measure, irrespective of the amount of structural information at $t$.

We call the intersection of information between the two hierarchies \(H\) and \(\hat{H}\) the \emph{structural agreement measure} at time \(t\):
\begin{equation}
    \alpha_H(\hat{H};t) = \int_{T^2} I_H(t,u,v)\cdot I_{\hat{H}}(t,u,v)\,\dif(u,v).
    \label{eq:local_agreement}
\end{equation}

The ratio between these two measures then gives us the portion of structural information at time $t$ on a reference hierarchy \(H\) is being recalled.
We call $\rho_H$ the local recall density of $\hat{H}$ on $H$:
\begin{equation}
    \rho_H(\hat{H};t) = \frac{\alpha_H(\hat{H};t)}{\iota(H;t)}.
    \label{eq:local_recall}
\end{equation}

Finally, we can define the overall recall as the average local recall density over time.
Notice that $\rho_H(\hat{H};t)$ is only defined for times where there is any information to recall at time $t$.
We define this informative subset of time span as 
\[
    \mathcal{T}_H = \{t \in T | \iota(H;t) > 0\}.
\]

This leads to the definition of L-measure:
\begin{align}
    \mathrm{L}_{\mathrm{recall}}(\hat{H} \mid H) 
    &= \frac{1}{|\mathcal{T}_H|}
    \int_{\mathcal{T}_H} \rho_H(\hat{H};t)\, \dif t, \nonumber \\
    \mathrm{L}_{\mathrm{precision}}(\hat{H} \mid H) 
    &= \frac{1}{|\mathcal{T}_{\hat{H}}|}
    \int_{\mathcal{T}_{\hat{H}}} \rho_{\hat{H}}(H;t)\, \dif t,
    \label{eq:new_lmeasure}
\end{align}
where \(|\mathcal{T}_H| = \int_{\mathcal{T}_H} \dif t\) is the total duration of informative time spans under $H$.

\subsection{Variations}
By using alternative definitions for time pair relevance induced by the hierarchy \(M\) and the significant triplet selection criteria \(I\), we can formulate the T-measure and its different variations.

\subsubsection{Full T-measure}
As mentioned, the T-measure can be seen as a variation of the L-measure, where the relevance mapping disregards label information.
If we label each segment in the hierarchy with a unique label, the full T-measure without windowing is equivalent to the L-measure.

In other words, we can express T-measure's relevance mapping as:
\begin{equation}
    M_H(t,u) = \max \big\{
        k \, | \, S^*_k(t) = S^*_k(u)
    \big\},
    \label{eq:tmeasure_relevance_map}
\end{equation}
where $S^*_k(t)$ identifies the segment containing $t$ at level $k$, as opposed to $S_k(t)$ which gives the label at time $t$.

\subsubsection{Reduced T-measure}
The reduced T-measure differs from the full T-measure by using an alternative definition for significant triplet indicator.
For reduced T-measure, the significant triplet are those where $u$, and $v$'s relevance to $t$ differ by exactly one level:
\begin{align}
    I'_H(t,u,v) = \big[ M_H(t,u) - M_H(t,v) = 1\big]_{\mathbbm{1}}.
    \label{eq:reduced_tmeasure}
\end{align}
This means that the reduced T-measure is only sensitive to refinements between consecutive levels of the hierarchy, as opposed to the overall ranking of all levels.

\subsubsection{Windowed T-measure}
The windowed T-measure extends the definition of $\iota_H(t)$ for using a time-varying window $W(t)$ around $t$ as the integration range, as opposed to integrating directly over the full time span.

Let $W(t) = \big( [t - w, t + w] \cap T\big)$ be a sliding window of size $\leq 2w$ around $t$.
We define the windowed structural information measure by changing the range of integration from the original:
\[
    \iota^W(H;t) = \int_{W^2(t)} I_H(t,u,v)\,\dif(u,v), 
    \label{eq:windowed_richness}
\]
\[ \displaystyle
    \alpha^W_H(\hat{H};t) = \int_{W^2(t)} I_H(t,u,v)\cdot I_{\hat{H}}(t,u,v)\,\dif(u,v) 
    \label{eq:windowed_agreement}
\]


The rest of the definitions follow.

\subsubsection{Monotonicity and the concept of relevance}
The relevance mapping \(M_H(u,v)\) for triplet-based metrics shown in ~\eqref{eq:relevance_mapping} assumes that H has monotonic labeling.
However, in practice, label monotonicity is often violated, and the current definition of \(M_H(u,v)\) can give rise to counterintuitive results.

Alternatively, we can formulate the relevance mapping with requirement for strict label monotonicity:
\begin{equation}
    M'_H(u,v) = \min \bigl\{k \, |\, S_i(u) \neq S_i(v)\bigr\} - 1,
    \label{eq:monotonic_meet}
\end{equation}
which measures the deepest level that two time points meet monotonicly by first finding the shallowest level where the two time points diverge in labeling.

Notice how for a monotonic hierarchy $H$, the two versions of the relevance mapping are equivalent: $M'_H(u,v) = M_H(u,v)$.  

In case where the hierarchy is not monotonic, the two definitions of relevance mapping can lead to different results.
Put in FIGURE that shows how M and M' for a non-monotnoic hierarchy.
While monotonicity is not a requirement for the triplet based metrics, we will see how using this substitute definition of relevance mapping can lead to different results when monotonicity is violated.
